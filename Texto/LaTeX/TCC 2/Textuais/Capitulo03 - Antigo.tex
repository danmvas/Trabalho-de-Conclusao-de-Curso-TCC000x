\chapter{Trabalhos relacionados}


{\color{red}estou ainda reescrevendo essa parte, já que vou ter de encontrar trabalhos novos}

Após uma busca pela literatura, foi encontrada uma tese de doutorado escrita por \citeonline{Thesis:2019}, intitulada ``\textit{Learning Analytics Integrating Student Attendance Data}'', que é uma pesquisa que explora a aplicação de técnicas de análise de dados para melhorar a compreensão do comportamento dos alunos em relação à frequência às aulas. A tese examina como os dados de frequência podem ser coletados, analisados e usados para melhorar o desempenho acadêmico dos alunos. A tese argumenta que a análise dos dados de frequência pode fornecer informações valiosas sobre o comportamento dos alunos e ajudar a identificar padrões que possam ser usados para melhorar o desempenho acadêmico, e apresenta um estudo empírico que usa técnicas de aprendizado de máquina para analisar os dados de frequência dos alunos, mostrando ser possível prever com precisão o desempenho acadêmico dos alunos com base em seus padrões de frequência às aulas.

Durante o trabalho, a autora utilizou a metodologia de ``\textit{data-first}'', abordagem que prioriza os dados como o ponto central em todo o processo de tomada de decisões e desenvolvimento de estratégias, para isolar e explicar a discrepância nos dados de frequência dos alunos. Em vez de tentar encontrar um evento conhecido no conjunto de dados, ela tentou identificar eventos candidatos dentro dos dados e, em seguida, explicá-los como uma atividade secundária. Essa abordagem permitiu que ela explorasse os padrões de frequência dos alunos e identificasse possíveis eventos que poderiam estar afetando a frequência. A metodologia utilizada pela autora baseou-se nos resultados do modelo preditivo. Ela usou esses resultados para responder à seguinte pergunta de pesquisa:  ``Esse modelo pode ser usado para fornecer informações adicionais sobre os padrões de frequência dos alunos?'', identificando métodos e ferramentas apropriados para análise numérica e estatística. Cada candidato precisava ser correlacionado com um evento acadêmico, social ou físico. Para realizar sua pesquisa, a autora utilizou uma abordagem mista que combinava métodos quantitativos e qualitativos, coletando dados quantitativos sobre a frequência dos alunos usando o sistema de gerenciamento de aprendizagem da universidade. Também utilizou técnicas estatísticas avançadas para analisar os dados coletados. Ela usou análise descritiva para descrever as características básicas dos dados, como média, mediana e desvio padrão, e também uma análise multivariada para examinar as relações entre as variáveis e identificar possíveis padrões.

A autora realizou uma vasta revisão da literatura em várias áreas, incluindo ciência da computação, ciência de dados, psicologia, filosofia moral e educação. Essa revisão permitiu que a autora identificasse as melhores práticas e tendências atuais em análise de dados educacionais. Durante a tese foi investigado separadamente cada uma das ferramentas/ambientes/sistemas mais populares ou mais citados, usando fontes como publicações acadêmicas, folhetos de dados públicos ou literatura de produtos/demonstrações. Isto permitiu que ela avaliasse as diferentes ferramentas disponíveis para análise de dados educacionais e determinasse quais seriam mais adequadas para sua pesquisa. No entanto, é importante notar que essas fontes não produzem uma avaliação completa em todas as situações, e a autora reconhece que essas fontes podem não fornecer uma avaliação completa das ferramentas em todas as situações e que pode haver outras ferramentas disponíveis que não foram incluídas em sua pesquisa.

Ao utilizar um modelo de aprendizado de máquina para identificar alunos que estavam em risco de se tornarem desengajados e faltosos e comparando os resultados do modelo com o método anteriormente utilizado pela universidade, a autora descobriu que o modelo de aprendizado de máquina identificou 62 alunos que não foram identificados pelo método anterior. Adicionalmente, a autora descobriu que a métrica de engajamento comportamental (BEM) era mais poderosa do que uma simples razão de sessões frequentadas, observando melhoria empírica de 12\% usando a métrica BEM em comparação com a razão de engajamento (ER). No entanto, ela não forneceu análise numérica e teoria matemática para explicar por que isso pode ser verdade. Por fim, também foi descoberto um padrão de grandes variações na contagem de alunos acima e abaixo do valor médio BEM em cada semana entre as semanas 5 e 8 do primeiro semestre. Ela usou essa observação para tentar isolar possíveis causas ou explicações para o efeito da falta de engajamento dos alunos.

% escrever mais algumas páginas
% - como que fez, com que técnicas, metodologia, forma de avaliação, resultados
% - figuras e esquemas sempre bom

Há também a dissertação de \citeonline{moissa2016}, que teve como o objetivo geral avaliar a influência das ferramentas de LA na interação, desempenho e satisfação dos alunos em ambientes virtuais de aprendizagem. Para alcançar esse objetivo, a autora propõe a realização de um experimento com usuários reais, a análise dos dados coletados e a comparação entre os alunos que tiveram acesso à ferramenta e os alunos que não tiveram, bem como entre os alunos que usaram a ferramenta e os que não usaram. Ainda que a autora teve foco específico em ambiente educacionais virtuais, o trabalho possui semelhança com o atual por buscar entender como as ferramentas de LA podem ser utilizadas para melhorar a experiência de aprendizagem de estudantes, compreendendo a influência dessas ferramentas para ajudar a desenvolver estratégias mais eficazes de ensino.

Para realizar a coleta dos dados, a autora desenvolveu ferramentas para auxiliar no processo de ensino-aprendizagem e realizou um experimento com usuários reais. Através desse experimento, foram coletados dados sobre a interação dos alunos com a ferramenta, seu desempenho e sua satisfação. Os dados foram analisados através de técnicas estatísticas e de Mineração de Dados, como análise de regressão, análise de correlação e análise de \textit{cluster}. Além disso, a autora aplicou um questionário de satisfação para coletar a opinião dos alunos em relação ao minicurso realizado e à ferramenta utilizada, sendo o questionário composto por questões objetivas, questões discursivas e questões de múltipla escolha.




%%%%%%%%%%%%%%%%%%%%%%%%%%%

\section{Análise Comparativa}


Nesta seção, será realizada uma análise comparativa entre os trabalhos relacionados e os dados coletados para a pesquisa de detecção de padrões em relação à presença dos estudantes no ensino básico brasileiro. Os trabalhos relacionados abordam diferentes aspectos da análise de dados educacionais e fornecem conclusões relevantes para a compreensão dos padrões de frequência dos alunos. Por mais que os trabalhos relacionados no exterior e no Brasil tenham grande enfoque na aplicação e otimização de sistemas de frequência, é importante que sejam estudados para que sejam percebidas quais as variáveis mais relevantes durante a coleta de frequência escolar. Sendo assim, é importante destacar que o trabalho atual não irá focar na aplicação de um novo sistema de frequência, e sim, em aplicar algoritmos de LA e EDM em dados que serão coletados futuramente, o que já se torna a diferença principal entre os trabalhos vistos nas seções \ref{sec:exterior} e \ref{sec:brasil}. Ao comparar esses trabalhos relacionados com a pesquisa proposta, observa-se que todos eles compartilham o objetivo comum de compreender e melhorar a frequência dos alunos no ensino básico. No entanto, cada estudo e país possui suas peculiaridades e desafios específicos, o que demanda abordagens adaptadas às suas realidades.

O estudo realizado por \citeonline{Thesis:2019} aborda a aplicação de técnicas de análise de dados para compreender o comportamento dos alunos em relação à frequência escolar. A autora utiliza técnicas de aprendizado de máquina para analisar os dados de frequência dos alunos e identificar padrões que possam contribuir para melhorar o desempenho acadêmico. A metodologia adotada incluiu a coleta de dados quantitativos sobre a frequência dos alunos e a aplicação de técnicas estatísticas avançadas para análise desses dados. Por outro lado, a dissertação de \citeonline{moissa2016} tem foco na avaliação do impacto das ferramentas de aprendizagem analítica (LA) na interação, desempenho e satisfação dos alunos em ambientes virtuais de aprendizagem. Embora o contexto seja diferente do estudo proposto, existe uma relevância em comum, que é entender como as ferramentas de LA podem ser utilizadas para aprimorar a experiência de aprendizagem dos estudantes. A autora conduziu um experimento com usuários reais, coletando dados sobre a interação dos alunos com a ferramenta e utilizando técnicas estatísticas e de Mineração de Dados para análise dos dados coletados. A abordagem mista, que combinou métodos quantitativos e qualitativos, proporcionou uma compreensão mais abrangente dos resultados obtidos. No entanto, o presente trabalho tem um foco distinto, concentrando-se em outras áreas, como a aplicação da detecção de padrões e a visualização das informações dos resultados obtidos. Busca-se explorar como essas técnicas podem ser utilizadas para identificar padrões de frequência escolar e visualizar esses dados de forma compreensível e significativa. Dessa forma, pretende-se oferecer visões aos educadores e formuladores de políticas educacionais, permitindo uma tomada de decisão embasada na compreensão dos padrões de frequência dos alunos. Assim, apesar das semelhanças entre os estudos mencionados, o atual trabalho busca aprimorar o conhecimento na área, concentrando-se em aspectos específicos da análise de dados e visualização das informações sobre a frequência escolar, com o objetivo de contribuir para a melhoria da experiência educacional dos alunos.

Na seção \ref{sec:exterior}, foram apresentadas iniciativas no exterior que abordam a detecção de frequência escolar por meio de tecnologias avançadas, como detecção facial, RFID e dispositivos vestíveis. Essas abordagens demonstram a diversidade de soluções utilizadas para coletar e analisar dados de frequência dos alunos. Os estudos apresentados destacam a precisão e eficiência das tecnologias utilizadas, além de considerar aspectos como segurança, portabilidade e disponibilidade dos dados. Já na seção \ref{sec:brasil}, foram abordadas iniciativas no Brasil, como o Cadastro Único para Programas Sociais (CadUnico) e o Sistema Presença. O CadUnico é uma fonte de dados importante para monitorar a escolaridade no país, registrando informações sobre a frequência dos alunos. O Sistema Presença, por sua vez, busca padronizar as informações sobre a frequência escolar em todo o país. Ele é utilizado pelos gestores educacionais para registrar a presença dos alunos nas escolas, permitindo o monitoramento em tempo real da frequência e a identificação de possíveis problemas de evasão escolar. O sistema também auxilia na tomada de decisões e na formulação de políticas públicas para garantir a universalização do acesso à educação, e o \textit{software} é o que servirá para aplicação do NEES.