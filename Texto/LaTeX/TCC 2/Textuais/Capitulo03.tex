\chapter{Trabalhos Relacionados}



Após uma busca pela literatura, foi encontrado o artigo de \citeonline{Lee2019}, intitulado \textit{``The Machine Learning-Based Dropout Early Warning System for Improving the Performance of Dropout Prediction''}. Este artigo tem como objetivo abordar a questão da previsão de evasão escolar usando técnicas de aprendizado de máquina, propondo o uso de métodos SMOTE e ensemble para lidar com esse problema. O SMOTE é uma técnica de \textit{oversampling} que gera novos exemplos sintéticos da classe minoritária (no caso do artigo, alunos em risco de evasão escolar), enquanto o ensemble combina vários classificadores para melhorar o desempenho geral. 

Os autores usaram dados do banco de dados do \textit{National Education Information System} (NEIS) para treinar e avaliar seus modelos de previsão. Eles incluíram os principais indicadores de risco recomendados para a predição da evasão escolar pelo National High School Center para a previsão de evasão escolar em sua análise, que foram a frequência de presença, comportamento e desempenho do curso (referidos como ``ABCs'': \textit{attendance}, \textit{behavior}, e \textit{course performance}). Para além desses, os autores, não tiveram acesso a muitos outros recursos, como a avaliação dos professores sobre os alunos.

Os autores testaram vários métodos de aprendizado de máquina para prever a evasão escolar, incluindo árvores de decisão, redes neurais, Support Vector Machine (SVM), Random Forest e Gradient Boosting Decision Tree (BDT). Para avaliar o desempenho dos modelos, os autores usaram curvas ROC e PR. As curvas ROC são usadas para avaliar a sensibilidade e especificidade do modelo em diferentes pontos de corte, enquanto as curvas PR são mais informativas quando há um desequilíbrio de classe. Apesar dessas limitações, os resultados mostram que o modelo BDT teve o melhor desempenho na previsão de alunos em risco de evasão escolar. De acordo com as curvas PR, o modelo BDT teve o melhor desempenho \cite{Lee2019}.

Já o trabalho de \citeonline{queiroga2020learning} apresenta um estudo de caso sobre o uso de análise de dados para identificar alunos em risco de evasão em um curso técnico de educação a distância. O objetivo do estudo foi desenvolver um modelo preditivo que pudesse identificar alunos em risco de evasão com antecedência, permitindo que a instituição de ensino tomasse medidas preventivas para ajudar esses alunos a permanecerem no curso. Para alcançar esse objetivo, os pesquisadores utilizaram técnicas de aprendizado de máquina, como algoritmos de classificação e regressão, para analisar dados de desempenho acadêmico, interações dos alunos com o ambiente virtual de aprendizagem e informações demográficas. Além disso, foram utilizadas técnicas de mineração de dados e análise exploratória de dados para identificar padrões e tendências nos dados. O estudo de \citeonline{queiroga2020learning} oferece uma análise detalhada sobre a identificação precoce de estudantes em risco de evasão em um curso técnico a distância no Brasil, utilizando uma abordagem analítica de aprendizado. Este estudo difere-se deste principalmente na metodologia empregada para análise dos dados, o artigo se debruça sobre a utilização de um algoritmo genético elitista para a afinação de hiperparâmetros, visando maximizar os resultados das previsões.

Ambos os estudos compartilham o objetivo comum de utilizar análise de dados para identificar estudantes em risco de evasão escolar. Contudo, as técnicas específicas e os contextos de aplicação diferem, o estudo de \citeonline{queiroga2020learning} foca em um curso técnico a distância específico, utilizando contagens de interações dos alunos com o ambiente virtual de aprendizagem como base para a previsão. A contribuição do artigo dos autores para o campo é notável na aplicação de algoritmos genéticos para a afinação de hiperparâmetros, uma abordagem inovadora que pode oferecer conhecimentos sobre a otimização de modelos preditivos em contextos educacionais. Isso complementa o presente trabalho ao demonstrar a eficácia de diferentes métodos analíticos e abordagens no domínio da análise educacional, enriquecendo assim a compreensão geral das estratégias possíveis para lidar com o problema da evasão escolar.

Já no artigo de \citeonline{Lee2019}, é apresentado um sistema de alerta precoce para evasão escolar baseado em aprendizado de máquina, visando melhorar a previsão de evasão escolar. Este estudo, realizado na Coréia do Sul, utiliza técnicas como Oversampling Sintético de Minoria (SMOTE) e métodos de ensemble para abordar o desequilíbrio de classes no conjunto de dados. Eles treinaram modelos como Random Forest e Árvores de Decisão Impulsionadas (Boosted Decision Trees) com e sem SMOTE, usando uma grande amostra de dados de estudantes do Sistema Nacional de Informações Educacionais (NEIS). O estudo revelou que a Árvore de Decisão Impulsionada apresentou o melhor desempenho, com base na análise das curvas ROC e PR. O artigo enfatiza a importância de lidar com o desequilíbrio de classes em modelos preditivos e destaca o uso de curvas PR como uma métrica de avaliação eficaz em conjuntos de dados desequilibrados.

Comparando com o presente trabalho, tanto o trabalho de \citeonline{Lee2019} quanto o de \citeonline{queiroga2020learning} os artigos abordam a questão da evasão escolar usando técnicas de aprendizado de máquina, mas com diferenças notáveis. Enquanto o artigo de \citeonline{Lee2019} foca em um sistema de alerta precoce utilizando uma grande base de dados nacional e abordagens específicas para lidar com o desequilíbrio de classes, este trabalho concentra-se na identificação de fatores de risco de evasão escolar no contexto educacional brasileiro, empregando uma abordagem mais diversificada de técnicas de aprendizado de máquina.

Já em relação ao estudo realizado no artigo de \citeonline{niyogisubizo2022predicting} onde também aborda o problema global do abandono escolar, ressaltando sua relevância não apenas para o indivíduo, mas para a sociedade como um todo. O artigo propõe um modelo inovador de \textit{ensemble stacking}, combinando Random Forest (RF), Extreme Gradient Boosting (XGBoost), Gradient Boosting (GB) e Feed-forward Neural Networks (FNN) para prever o abandono de estudantes em classes universitárias. Este modelo demonstrou um desempenho superior em relação aos modelos base, utilizando métricas de precisão de teste e área sob a curva (AUC) sob as mesmas condições. O modelo proposto no artigo se destaca por ser um dos primeiros a explorar o \textit{ensemble stacking} entre RF, XGBoost, GB e FNN para previsão de abandono escolar em classes universitárias.

O artigo apresenta uma abordagem de validação cruzada durante o treinamento do modelo para evitar o overfitting. Os dados de entrada foram divididos aleatoriamente em conjuntos de treinamento e teste, com porcentagens de 80\% e 20\%, respectivamente, e aplicou-se validação cruzada de dez vezes. Os resultados experimentais do modelo proposto foram comparados com os modelos base no primeiro layer, mostrando que o modelo de \textit{ensemble stacking} obteve desempenho superior em precisão de treinamento, variando de 86,67\% a 96,67\%, e precisão de teste, variando de 76,67\% a 92,18\%.

O modelo de \textit{ensemble stacking} atingiu resultados de recall de 0,93 e precisão de 0,93 nos testes, indicando que o modelo pode prever corretamente cerca de 93\% dos casos de abandono escolar. Além disso, alcançou um F1-Score geral de 0,92, demonstrando um desempenho de previsão eficiente para abandono escolar em classes universitárias. Este resultado é reforçado pelos resultados da AUC e pelas curvas ROC correspondentes para FNN, RF, GB, XGBoost e o \textit{ensemble stacking}, com o último alcançando uma pontuação AUC de 0,983, indicando uma classificação aceitável e superior à adivinhação aleatória. Em comparação com o presente trabalho, que foca na evasão escolar no contexto do sistema educacional brasileiro, utilizando algoritmos classificatórios e análises estatísticas, esse artigo fornece uma abordagem complementar, empregando um modelo híbrido de \textit{ensemble stacking} com técnicas avançadas de machine learning para prever o abandono escolar em um contexto universitário. O artigo traz uma nova perspectiva com o uso de um conjunto diversificado de algoritmos de machine learning em um contexto mais global.


\begin{table}[ht!]
\centering
\caption{Comparação entre os estudos de evasão escolar}
\label{tab:comparacao_estudos}
\resizebox{\textwidth}{!}{%
\begin{tabular}{|l|l|l|l|l|}
\hline
\textbf{Critério} &
  \textbf{Contexto do Estudo} &
  \textbf{Métodos Utilizados} &
  \textbf{Abordagem de Análise} &
  \textbf{Principais Resultados} \\ \hline
\textbf{\citeonline{queiroga2020learning}} &
  Curso técnico a distância &
  Algoritmo genético &
  Análise de interações &
  Otimização de hiperparâmetros \\ \hline
\textbf{\citeonline{Lee2019}} &
  Universidade na Coreia do Sul &
  \textit{Ensemble stacking} &
  Análise baseada em big data &
  Previsão de abandono escolar \\ \hline
\textbf{\citeonline{niyogisubizo2022predicting}} &
  Universidade em Nitra &
  \textit{Ensemble stacking} &
  Validação cruzada &
  Alta precisão de previsão \\ \hline
\textbf{Presente trabalho} &
  Sistema educacional brasileiro &
  \begin{tabular}[c]{@{}l@{}}Random Forest, Decision Tree,\\ Naive Bayes, AdaBoost, KNN\end{tabular} &
  Análise classificatória &
  Identificação de fatores de risco \\ \hline
\end{tabular}%
}
\fonte{\me{2023}}
\end{table}

O primeiro artigo de \citeonline{queiroga2020learning}, focado em um curso técnico a distância no Brasil, utiliza um algoritmo genético para a afinação de hiperparâmetros, destacando-se por sua abordagem direcionada ao aprimoramento de modelos preditivos em um ambiente educacional específico. Este estudo realça a importância de ajustar minuciosamente os modelos de machine learning para refletir com precisão as peculiaridades e necessidades específicas de um contexto educacional particular. Já o segundo artigo de \citeonline{Lee2019}, originário de uma universidade em Nitra, propõe um modelo similar de \textit{ensemble stacking}, mas se distingue pelo uso de validação cruzada para evitar o overfitting, garantindo que o modelo seja robusto e confiável. Este estudo ressalta a importância de metodologias rigorosas no treinamento de modelos de machine learning, especialmente em contextos onde os dados podem ser limitados ou desbalanceados. Por fim, o terceiro artigo de \citeonline{niyogisubizo2022predicting}, apresenta um modelo de \textit{ensemble stacking}, combinando várias técnicas avançadas de machine learning, como Random Forest, Extreme Gradient Boosting, Gradient Boosting e Feed-forward Neural Networks. Esta pesquisa é pioneira no uso dessa abordagem de ensemble para prever a evasão escolar em classes universitárias, demonstrando a eficácia de combinar diferentes métodos para melhorar a precisão das previsões. Ao comparar estes três estudos, é evidente que, embora cada um adote uma abordagem única, todos reconhecem a complexidade da evasão escolar e a necessidade de técnicas sofisticadas de análise de dados para abordá-la efetivamente.

Tendo conhecimento dos trabalhos relacionados, analisaremos no capítulo seguinte o contexto da pesquisa realizada, bem como entrar em detalhamento sobre a base de dados fornecida.










% O trabalho de Lee e Chung contribui significativamente para o campo de previsão de evasão escolar ao demonstrar a eficácia de técnicas específicas em um conjunto de dados desequilibrado e ao destacar a importância das curvas PR na avaliação de modelos em tais contextos. Este artigo serve como um importante ponto de referência para o trabalho de conclusão de curso, oferecendo percepções sobre o tratamento de desequilíbrios de classe e a eficácia de diferentes técnicas de aprendizado de máquina, que podem ser aplicáveis ao contexto brasileiro.



