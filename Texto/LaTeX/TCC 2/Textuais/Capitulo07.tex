\chapter{Conclusão}



Este trabalho de conclusão de curso se propôs a investigar a evasão escolar no contexto do Sistema Alerta Preventivo (SAP), empregando técnicas de classificação de dados para identificar quais foram os fatores de risco. A pesquisa foi embasada em levantamento bibliográfico e na aplicação de algoritmos de reconhecimento de padrões em mineração de dados, utilizando o Instrumento de Avaliação dos Fatores de Risco de Evasão Escolar (IAFREE), desenvolvido pela TED 10974 do NEES.

O trabalho iniciou com a proposta de identificar padrões em uma base do sistema de frequência pertencente à TED 11476, que pretende reformular o atual Sistema Presença em vigor no Brasil, que coleta a frequência escolar de estudantes do ensino básico que estão associados ao programa Bolsa Familia \cite{BolsaFamilia}. Para isso, foi estudada a arquitetura do sistema disponibilizada pelo NEES, sistemas de coleta de frequência na literatura nacional e internacional e algoritmos de reconhecimento de padrões baseados em \textit{machine learning}, na expectativa que os dados recebidos pudessem já ser aplicados à arquitetura que será usada pela TED 11476 no futuro. Como não foi possível receber os dados da TED do Sistema Presença, foi conversado com a TED 10974 para análise dos dados e posteriormente foram os que realmente chegaram a ser utilizados para o trabalho.

Todavia, mesmo que os dados com relação à frequência sobre os estudantes não foram recebidos, os estudos sobre sistemas de frequência no Brasil e no mundo foi compilado, aceito e apresentado no \textit{II Workshop de Aplicações Práticas de Learning Analytics em Instituições de Ensino no Brasil}. O evento ocorreu em outubro de 2023, e possui objetivos de facilitar a interação entre pesquisadores e profissionais, visando compartilhar experiências práticas sobre a adoção e implementação de aprendizagem de máquina no Brasil, apresentando ferramentas, estratégias e abordagens eficazes para essa adoção. O título do trabalho é ``Sistemas de frequência escolar: avanços, desafios e atuais perspectivas'' \cite{wapladan}.


Para se compreender quais algoritmos deveriam ser utilizados e atingir o primeiro objetivo específico da pesquisa, foi realizada uma pesquisa sobre as áreas de \textit{Educational Data Mining} (EDM) e \textit{Learning Analytics} (LA), enfatizando a importância do reconhecimento de padrões e das considerações éticas na coleta e análise de dados educacionais. O foco foi entender como os dados podem ser utilizados para melhorar a educação, considerando aspectos éticos, legais e de privacidade. Foi decidido que os algoritmos a serem utilizados seriam: Decision Tree, Random Forest, Naive Bayes, AdaBoost e K Vizinhos Próximos (KNN), que são algoritmos clássicos da literatura e representam abordagens variadas em aprendizado de máquina. Para comparação da perfomance dos modelos, foram utilizadas as métricas de acurácia, precisão, recall score e f1 score. Também foram feitas as matrizes de confusão de cada algoritmo para melhor visualização dos resultados, inseridas nos Apêndices deste trabalho.

Tendo decidido então pela aplicação de reconhecimento de padrões não mais na base de frequência no novo Sistema Presença, mas sim na do SAP, foi feito todo um estudo dos dados disponibilizados, atingindo o segundo objetivo específico da pesquisa. A base continha respostas de 17710 crianças do ensino fundamental, variando do primeiro ao nono ano, coletadas em 2022, vindas de todas as regiões do Brasil. Os estudantes responderam um questionário socioeconômico e 36 perguntas relacionadas a desejos de evasão, variando de 1 a 7 em uma escala de Likert. Essas perguntas, para análise, foram classificadas entre dimensões e fatores para melhor compreensão de possíveis motivos que pudessem levar à evasão escolar. Esse questionário, na íntegra, encontra-se no Anexo 1 deste trabalho.

Para fazer as análises estatísticas e atingir o terceiro objetivo específico da pesquisa, foi utilizado p-valores para avaliar a significância estatística das relações entre variáveis socioeconômicas e dimensões educacionais. Entre as dimensões analisadas, E\_ESCV (Estudante-Escola), E\_PROFV (Estudante-Profissionais da Escola), E\_FAMV (Estudante-Família) e E\_COMV (Estudante-Comunidade) mostraram variações significativas em suas associações com as variáveis socioeconômicas. Por exemplo, o porte da escola impacta as dimensões relacionadas à escola e à comunidade, enquanto o sexo dos estudantes influencia dimensões como E\_PROFV, E\_FAMV e E\_COMV.

Por fim, atingindo o quarto objetivo específico, a análise classificatória realizada revelou nuances importantes nos diferentes modelos avaliados, cada um com seus pontos fortes e limitações específicas. Esta pesquisa destacou a capacidade de modelos como Decision Tree e Random Forest em fornecer previsões confiáveis e detalhadas sobre variáveis educacionais críticas. Da mesma forma, a análise revelou a necessidade de customização de modelos como o AdaBoost para atender às necessidades específicas do contexto educacional. A seleção do modelo ideal depende do contexto específico de aplicação, da natureza dos dados e dos objetivos da análise. Portanto, o entendimento detalhado de cada modelo e suas características foi essencial para a seleção e otimização apropriadas no contexto educacional da base recebida. A análise classificatória com modelos de \textit{machine learning} (Decision Tree, Random Forest, AdaBoost, Naive Bayes Categórico e K Vizinhos Próximos) ofereceu visões sobre a eficácia de cada algoritmo. Decision Tree e Random Forest apresentaram um desempenho superior, enquanto AdaBoost e Naive Bayes Categórico mostraram resultados abaixo do esperado, sendo que K Vizinhos Próximos teve um desempenho mediano.

Sendo assim, unindo ambas as análises, é possível concluir que os fatores de risco na base do IAFREE que possuem maior influência estatística ou melhor performance entre os algoritmos de classificação, são E\_FAM1V (Suporte Familiar) e E\_COM3V (Distanciamento escola – comunidade) para os fatores, e E\_FAMV (Estudante-Família) para as dimensões.

Como trabalhos futuros, há a possibilidade de análises dos dados das perguntas individuais, sem associação a dimensão ou fatores, bem como um estudo social para a compreensão de possíveis correlações e motivações por detrás de respostas com valores mais altos para determinadas perguntas. Também é possível estudar possíveis medidas de retenção escolar focando em determinadas dimensões e fatores.

No mundo em rápida mudança da educação e da tecnologia, a capacidade de adaptar e refinar continuamente os modelos de aprendizagem é crucial. A abordagem adotada aqui fornece uma base sólida para futuras pesquisas e desenvolvimentos na área mais ampla de \textit{analytics} preditivos na educação. Este trabalho deseja contribuir para uma compreensão mais profunda de como a tecnologia pode ser usada para enriquecer e aprimorar a experiência educacional para alunos e educadores.



% \begin{itemize}
%     \item mostrar processo do tcc
%     \item mostrar como os objetivos foram alcançados
%     \item explorar resultados
%     \item fazer uma discussão
%     \item falar da publicação WAPLA
% \end{itemize}